\documentclass[12pt]{alittlebear}

\def\name{Joseph Siu}
\def\course{MAT159: Analysis II}
\def\headername{Homework }
\def\headernum{4}

\begin{document} 

\newl{l1}{
    Lebesgue Integrability implies Riemann Integrability.

    \newp{[Proof of Lemma \ref{lemma:l1}]\hfill
        
        Fix $\ep>0$, let $D:=\{x\in[a,b]\mid \T{ $f$ is discontinuous at $x$}\}$, let $E:=\{x\in[a,b]\mid V_f(x)\geq \ep\}$. Then, we have $E\subseteq D$. Since $f$ is continuous almost everywhere, we have $D$ is a null set, thus $E$ is also a null set.

        Since $E$ is closed (the complement is open) and bounded, it is compact, thus there exists a finite subcover of $E$: $I=\{I_1,\ldots,I_n\}$. By definition of null set we are able to let $I_i$ be sufficiently small such that $\sum_{i=1}^n V_f(I_i)|I_i|<\frac{\ep}{4}$ because of our $|I_i|$ here.

        Now, the set $A=[a,b]\setminus \cup_{i=1}^n I_i$ is also compact since it is bounded and $I$ is an open cover. Thus, there also exists a finite subcover of $A$: $J=\{J_1,\ldots,J_m\}$. Since for all $a\in A$, by our construction of $E$ we have $V_f(a)<\ep$, thus by choosing smaller $\ep$ we may assume $\sum_{i=1}^m V_f(J_i)\abs{J_i}<\frac{\ep}{4}$.

        Since $[a,b]$ is covered by the union of $I$ and $J$, we have when the norm of the arbitrary marked partition $(P,\eta)$ is sufficiently small, then 
        \begin{align*}
            \sum_{i=0}^{n-1}V_f([x_i,x_{i+1}])\Delta x_i&\leq\sum_{i=1}^n V_f(I_i)\abs{I_i}+\sum_{i=1}^m V_f(J_i)\abs{J_i}\\
            &<\frac{\ep}{4}+\frac{\ep}{4}\\
            &=\frac{\ep}{2}\\
            &<\ep.
        \end{align*}

        Thus, since $\ep$ is arbitrary, we have shown the Riemann integrability of $f$.
    }
}

\newpage

\newe{e1}{
    Determine if each of the function is Riemann integrable on the given interval on $[0,1]$. Prove your statement.

    \newq{q1}{
        \[D(x)=\begin{cases}
            1 & x\in\Q,\\
            0 & x\notin\Q.
        \end{cases}\]
        \qbreak
        \newcl{cl1}{
            The Dirichlet function is not Riemann integrable.
        }
        \newp{
            We want to show $$ \forall \al\in\R.\exists \ep >0.\forall \delta >0. \T{ s.t. }\exists (\Gamma, \eta ) \T{ of }[0,1]. \norm{\Gamma}<\delta\land \abs{\sigma(D, \Gamma, \eta)-\al}\geq\ep.$$
            
            To this end, let $\al\in\R$ be arbitrary. Fix $\ep=\frac14>0$, and let $\delta\in\R^+$ be arbitrary. 

            Now, consider 2 different $\Gamma$:
            \begin{enumerate}
                \item For $\Gamma_1$, we choose all $\eta$ to be rational numbers;
                \item For $\Gamma_2$, we choose all $\eta$ to be irrational numbers;
            \end{enumerate}
            Then, we can see that $\sigma(D,\Gamma_1,\eta)=1$ and $\sigma(D,\Gamma_2,\eta)=0$, for any $\al$, at least one of these 2 the differences is larger than $\ep$. Formally, if $\al\geq\frac12$, we use $\sigma(D,\Gamma_2,\eta)=0$, so the differences is always at least $\frac12$; if $\al<\frac12$, we use $\sigma(D,\Gamma_1,\eta)=1$, so the differences is always at least $\frac12$, and since $\frac12>\frac14=\ep$, we conclude the negation of Riemann integrable holds, this implies the Dirichlet function is not Riemann integrable, as needed.

            Alternately, we can see the function is continuous nowhere, thus it is not Lebesgue Integrable (the discontinuity points do not form a null set), thus it is not Riemann Integrable.
        }
    }

    \newpage
    \newq{q2}{
        \[T(x)=\begin{cases}
            \frac1q, & x=\frac{p}{q}\in\Q, p\in\Z, q\in\N, (p,q)=1,
            \\0, & x\notin\Q.
        \end{cases}\]
        \qbreak
        \newcl{cl2}{
            The Riemann-Thomae function is Riemann Integrable. 
        }
        \newp{
            By MAT157 Homework, we have shown that the Riemann-Thomae function is discontinuous whenever $x\in\Q$ and continuous whenever $x\notin\Q$. Now we show that $T(x)$ is continuous almost everywhere on $[0,1]$ which implies that $T(x)$ is Riemann integrable on $[0,1]$.

            To this end, we show $\Q\cap[0,1]$ is a null set.
            
            Since $\Q$ is countable, so is $\Q\cap[0,1]$ (since the intersection is infinite), thus we are able to list the elements of $\Q\cap[0,1]$ as $$\{x_1,x_2,\dots\}\subseteq\bigcup_{i=1}^\infty I_{\frac{\ep}{2^{i+3}}}(x_i).$$ By the geometric series formula, we have $$\sum_{i=1}^\infty\abs{I_{\frac{\ep}{2^{i+3}}}(x_i)}=\sum_{i=1}^\infty \ep\frac{1}{2^{i+2}}=\frac{\ep}{4}\sum_{i=1}^\infty \frac{1}{2^{i}}=\frac{\ep}{2}<\ep.$$

            Hence, by definition this shows that $\Q\cap[0,1]$ is a null set, and since $T(x)$ is continuous almost everywhere on $[0,1]$, we conclude that $T(x)$ is Riemann integrable on $[0,1]$, which completes our proof.
        }
    }

    \newpage
    \newq{q3}{
        \[H(x)=\begin{cases}
            \frac1x-\sqrbra{\frac1x}, & x\neq0,\\
            0, & x=0.
        \end{cases}\]
        \qbreak
        \newcl{cl3}{
            $H(x)$ is Riemann Integrable.
        }
        \newp{
            The only discontinuous points of $H(x)$ on $[0,1]$ are $x=0$ or $x=\frac1n$ for $n\in\N$, thus we can see there are countably many discontinuous points, which implies that $H(x)$ is continuous almost everywhere on $[0,1]$ (except countably many points). Thus, we conclude $H(x)$ is Riemann integrable on $[0,1]$.
        }
    }

    \newpage
    \newq{q4}{
        \[G(x)=\begin{cases}
            \sgn\bra{\sin\frac{\pi}{x}}, & x\neq0,\\
            0, & x=0.
        \end{cases}\]
        \qbreak
        \newcl{cl4}{
            $G(x)$ is Riemann Integrable.
        }
        \newp{
            We can see the only values that $\sin(\frac{\pi}{x})$ changes sign are when $x=\frac{1}{n}$ for some $n\in\N$. Moreover, from MAT157 we have proven that $G(x)$ has second type of discontinuity at $x=0$. Thus, since $\{x\in[0,1]\mid x=\frac1n, n\in\N\}\cup\{0\}$ is a countable set, we have that $G(x)$ is continuous almost everywhere on $[0,1]$ (except countably many points), which implies that $G(x)$ is Riemann integrable on $[0,1]$.
        }
    }

    \newpage
    \newq{q5}{
        \[K(x)=\begin{cases}
            \sin\bra{\ln\frac1x}, & x\neq0,\\
            0, & x=0.
        \end{cases}\]
        \qbreak
        \newcl{cl5}{
            $K(x)$ is Riemann integrable.
        }
        \newp{
            Since $\ln\frac1x$ is continuous on $(0,1]$ but not  $x=0$, and $\sin(x)$ is continuous everywhere, we conclude that $K(x)$ is continuous almost everywhere on $[0,1]$ by MAT157 Homework (at most not continuous only at $x=0$ which is at most a null set). Thus, we conclude that $K(x)$ is Riemann integrable on $[0,1]$.

            Speficially, for any $ x\in(0,1]$ we can always find a sufficiently small closed neighborhood of $x$ such that the function is continuous on the neighborhood, which implies the only possible discontinuity of $\sin(\ln(\frac1x))$ is at $x=0$, which is a null set.
        }
    }
}

\end{document}  