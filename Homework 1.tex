\documentclass{homework}
\author{Joseph Siu}
\class{MAT159 - Analysis II}
\date{\today}
\title{Homework 1}

\newcommand{\Set}[1]{\{#1\}}
\newcommand{\T}[1]{\text{#1}}
\newcommand{\Al}[3]{#1 &=#2 &\text{#3}&&\\}

% Symbols
\newcommand*{\eg}{\leavevmode\unskip , e. g., \ignorespaces} % for example
\newcommand*{\ie}{\leavevmode\unskip, i. e., \ignorespaces} % that is
\newcommand{\nil}{\varnothing}
\AtBeginDocument{\def\O{\cal{O}}} % Big Oh
\AtBeginDocument{\def\C{\bb{C}}} % Complex
\newcommand{\R}{\bb{R}} % Reals
\newcommand{\Q}{\bb{Q}} % Rationals
\newcommand{\Z}{\bb{Z}} % Integers
\newcommand{\N}{\bb{N}} % Naturals
\renewcommand{\P}{\bb{P}} % Primes
\newcommand{\Pset}[1]{\mathcal{P}(#1)} %power set
\newcommand{\Relate}[2]{#1\mathcal{R}#2} %relation
\newcommand{\relate}{\mathcal{R}}
\newcommand{\F}{\bb{F}} 
\newcommand{\GF}[1][2]{\bb{F}_{#1}} 
\newcommand{\modulo}[1][n]{\Z/#1\Z} 
\newcommand{\ra}{\rightarrow}
\newcommand{\Ra}{\Rightarrow}
\newcommand{\?}{\stackrel{?}{=}}
\newcommand{\is}{\equiv}
\newcommand{\al}{\alpha}
\newcommand{\ep}{\varepsilon}
\renewcommand{\phi}{\varphi}
\newcommand{\p}{\partial}
\newcommand{\injective}{\hookrightarrow}
\newcommand{\surjective}{\twoheadrightarrow}
\newcommand{\bijective}{\hookrightarrow\mathrel{\mspace{-15mu}}\rightarrow}
\newcommand{\derivative}[2][x]{\frac{\D #2}{\D #1}}
\newcommand{\ceil}[1]{\left\lceil#1\right\rceil}
\newcommand{\floor}[1]{\left\lfloor#1\right\rfloor}
\newcommand{\near}[1]{\left\lfloor#1\right\rceil}
\newcommand{\arr}[1]{\left\langle#1\right\rangle}
\newcommand{\paren}[1]{\left(#1\right)} %pair / ()
\newcommand{\brk}[1]{\left[#1\right]} %[]
\newcommand{\abs}[1]{\left|#1\right|}
\newcommand{\curl}[1]{\left\{#1\right\}} %set {}
\newcommand{\func}[3]{#1: #2 \rightarrow #3}


\theoremstyle{definition}
\newtheorem*{claim}{Claim}
\newtheorem{definition}{Definition}
\newtheorem{theorem}{Theorem}
\newtheorem{lemma}{Lemma}

\begin{document} \maketitle

\section*{EXERCISE 1}
This exercise aims at computing the following indefinite integral

$$J_{n}=\int \frac{1}{\left(x^{2}+a^{2}\right)^{n}}\D x, n \in \mathbb{N}$$

\question Compute $J_{1}$.

\begin{align*}
    J_1&=\int \frac{\D x}{x^2+a^2}\\
    &=\int \frac1{a^2}\cdot\frac{\D x}{(\frac{x}a)^2+1}\\
    &=\frac1a \int \frac{\D (\frac{x}a)}{(\frac{x}a)^2+1}\\
    &=\frac1a\arctan(\frac{x}a)+C
\end{align*}
\newpage

\question Compute $J_{2}$. (Hint: Consider the substitution $x=\tan t$.)

\begin{align*}
    J_2 &=\int \frac{\D x}{(x^2+a^2)^2}\\
    \text{Substitute } x&=\tan t\\
    \D x&=\sec^2 t \cdot \D t\\
    J_2&=\int \frac{\sec^2 t}{(\tan^2 t + a^2)^2}\D t\\
    \D (\frac{1}{\tan^2 t + a^2}) &= -\frac{2\tan t\sec^2 t}{(\tan^2 t + a^2)}\D t\\
    J_2&=-\int \frac{1}{2\tan t}\D (\frac{1}{\tan^2 t + a^2})\\
    &=-\int \frac{1}{2x}\D (\frac{1}{x^2+a^2})\\
    \text{By integration by parts, we have }&\\
    J_2&=-\frac{1}{2x(x^2+a^2)}+\int \frac{1}{x^2+a^2}\D (\frac{1}{2x})\\
    &=-\frac{1}{2x(x^2+a^2)}+\frac{1}{2}\int \frac{1}{x^2+a^2}\D (\frac{1}{x})\\
    &=-\frac{1}{2x(x^2+a^2)}-\frac{1}{2}\int \frac{1}{x^2+a^2}\frac{1}{x^2}\D x\\
    \text{By partial fractions, we have}\\
    J_2&=-\frac{1}{2x(x^2+a^2)}-\frac{1}{2}\int -\frac{1}{a^2(x^2+a^2)}+\frac{1}{a^2x^2}\D x\\
    &=-\frac{1}{2x(x^2+a^2)}+\frac{1}{2}\int \frac{1}{a^2(x^2+a^2)}\D x -\frac{1}{2}\int\frac{1}{a^2x^2}\D x\\
    &=-\frac{1}{2x(x^2+a^2)}+\frac{1}{2a^2}\int \frac{1}{x^2+a^2}\D x -\frac{1}{2a^2}\int\frac{1}{x^2}\D x\\
    &=-\frac{1}{2x(x^2+a^2)}+\frac{1}{2a^3}\arctan(\frac{x}a) +\frac{1}{2a^2 x} + C\\
    &=\frac{x^2+a^2-a^2}{2xa^2(x^2+a^2)}+\frac{1}{2a^3}\arctan(\frac{x}a) + C\\
    &=\frac{x}{2a^2(x^2+a^2)}+\frac{1}{2a^3}\arctan(\frac{x}a) + C\\
\end{align*}
\newpage
\question Prove that $\forall n \in \mathbb{N}$,

$$
J_{n}=\frac{x}{\left(x^{2}+a^{2}\right)^{n}}+2 n \int \frac{x^{2}}{\left(x^{2}+a^{2}\right)^{n+1}} d x
$$

(Hint: Consider integration by parts.)

\begin{proof}
    Pick an arbitrary $n\in\N$, we have that
    \begin{align*}
    \D (\frac{1}{(x^2+a^2)^{n}}) &= -\frac{2nx}{(x^2+a^2)^{n+1}}\D x\\
        \frac{x}{(x^2+a^2)^{n}}+2n\int \frac{x^2}{(x^2+a^2)^{n+1}}\D x &= \frac{x}{(x^2+a^2)^{n}}+2n\int -\frac{x}{2n}\D (\frac{1}{(x^2+a^2)^{n}})\\
        &=\frac{x}{(x^2+a^2)^{n}}-\int x\D (\frac{1}{(x^2+a^2)^{n}})\\
        \text{By integration by parts, }\\
        \frac{x}{(x^2+a^2)^{n}}-\int x\D (\frac{1}{(x^2+a^2)^{n}}) &= \frac{x}{(x^2+a^2)^{n}} - \frac{x}{(x^2+a^2)^{n}} + \int \frac{1}{(x^2+a^2)^{n}}\D x\\
        &= \int \frac{1}{(x^2+a^2)^{n}}\D x\\
        &= J_n
    \end{align*}
    Thus we have shown for any $n\in\N$ the equality holds, completing our proof.
\end{proof}

\newpage

\question Based on the previous sub-question, establish the recursive relation:

$$
J_{n+1}=\frac{2 n-1}{2 n} \frac{1}{a^{2}} J_{n}+\frac{1}{2 n a^{2}} \frac{x}{\left(x^{2}+a^{2}\right)^{n}}
$$

From there, conclude that $J_{n}$ is an integral of finite terms (i.e., it is an elementary function) 

\begin{align*}
    J_{n+1}&=\int \frac{1}{(x^2+a^2)^{n+1}}\D x\\
    &= \frac1{a^2}\int \frac{(x^2+a^2)-x^2}{(x^2+a^2)^{n+1}}\D x\\
    &=\frac1{a^2}\int\frac1{(x^2+a^2)^n}\D x - \frac1{a^2}\int\frac{x^2}{(x^2+a^2)^{n+1}}\D x\\
    &=\frac1{a^2}J_n - \frac1{a^2}\int -\frac{x}{2n}\D (\frac{1}{(x^2+a^2)^n})\\
    &=\frac1{a^2}J_n + \frac1{2na^2}\int x\D (\frac{1}{(x^2+a^2)^n})\\
    &=\frac1{a^2}J_n + \frac{1}{2na^2}\left[\frac{x}{(x^2+a^2)^n}-\int\frac1{(x^2+a^2)^n}\D x \right]\\
    &=\frac1{a^2}J_n + \frac1{2na^2}\frac{x}{(x^2+a^2)^n}-\frac{1}{2na^2}J_n\\
    &=J_n[\frac{1}{a^2}-\frac{1}{2na^2}]+\frac{1}{2na^2}\frac{x}{(x^2+a^2)^n}\\
    &=(1-\frac1{2n})\frac1{a^2}J_n + \frac1{2na^2}\frac{x}{(x^2+a^2)^n}\\
    &=\frac{2n-1}{2n}\frac1{a^2}J_n + \frac1{2na^2}\frac{x}{(x^2+a^2)^n}
\end{align*}

as needed.

\newpage
\question Compute $J_{3}$.

By Q4's formula, combining with our result of $J_2$, we have that

\begin{align*}
    J_3&=\frac{4-1}{4}\frac{1}{a^2}J_2+\frac{1}{4a^2}\frac{x}{(x^2+a^2)^2}\\
    &=\frac{4-1}{4}\frac{1}{a^2}[\frac{x}{2a^2(x^2+a^2)}+\frac{1}{2a^3}\arctan(\frac{x}a) + C]+\frac{1}{4a^2}\frac{x}{(x^2+a^2)^2}\\
    &=\frac{3}{4a^2}\frac{x}{2a^2(x^2+a^2)}+\frac{3}{4a^2}\frac{1}{2a^3}\arctan(\frac{x}a)+\frac{1}{4a^2}\frac{x}{(x^2+a^2)^2} + C\\
    &=\frac{1}{4a^2}\left[\frac{3}{2a^3}\arctan(\frac{x}a)+\frac{3x}{2a^2(x^2+a^2)}+\frac{x}{(x^2+a^2)^2}\right] + C\\
\end{align*}

\end{document}