\documentclass[reqno]{alittlebear}
\def\filenum{3}

\begin{document}

\begin{exercise}{}{}
    \begin{note}
        The objective of this exercise is to completely solve integrals of the tyle \[\int\sin^\alpha x\cos^\beta x\D x,\quad x\in\left(0,\frac{\pi}{2}\right),\alpha,\beta\in\mathbb{Z}.\]
    \end{note}

    \begin{question}{}{q1}
        As a warm-up, compute the following integrals (2.-9.) from the above family. In each case, indicate what is the value of $\alpha$ and $\beta$ as well.\\

        Example.
        \begin{enumerate}
            \begin{mathnote}
                \item[1.] Since $1=\sin^0x\cos^0x$, we have $\alpha=0,\beta=0$. And, \[\int 1\D x = x + C\numberthis.\]
            \end{mathnote}
        \end{enumerate}
        \qbreak
        \begin{enumerate}
        \begin{mathnote}
                \item[2.]  Since $\cos x = \sin ^0 x\cos^1 x$, we have $\alpha=0,\beta=1$, and \[\int\cos x\D x = \sin x + C.\numberthis\]
                \item[3.] Since $\frac{1}{\cos x} = \sin ^0 x \cos ^{-1} x$, we have $\alpha=0,\beta=-1$. From Homework 2 Question 11 Claim 2, we have proven that \begin{align*}
                    \int \frac{1}{\cos x}\D x &= -\ln\abs{\csc\bra{\frac{\pi}{2}-x}-\cot\bra{\frac{\pi}{2}-x}}+C\\
                &=\ln\abs{\frac{1}{\frac{1}{\cos x}-\frac{\sin x}{\cos x}}} + C\\
                &=\ln\abs{\frac{\cos x}{1-\sin x}\cd\frac{1+\sin x}{1+\sin x}} + C\\
                &=\ln\abs{\frac{\cos x(1+\sin x)}{1-\sin^2 x}} + C\\
                &=\ln\abs{\frac{\cos x(1+\sin x)}{\cos^2 x}} + C\\
                &=\ln\abs{\frac{1+\sin x}{\cos x}} + C\\
                &=\ln\abs{\sec x + \tan x} + C.\numberthis\label{eq3}
                \end{align*}
            \end{mathnote}
            \begin{mathnote}
                \item[4.] Since $\sin x=\sin^{1}x\cos^{0}x$, we have $\alpha=1,\beta=0$. And, \[\int\sin x\D x = -\cos x + C.\numberthis\]
                \item[5.] Since $\sin x\cos x=\sin^{1}x\cos^{1}x$, we have $\alpha=1,\beta=1$. And, \begin{align*}
                    \int \sin x \cos x \D x &= -\int \cos x \D\bra{\cos x}\\
                &= -\frac{1}{2}\cos^2 x + C.\numberthis
                \end{align*}
                \item[6.] Since $\frac{\sin x}{\cos x}=\sin^{1}x\cos^{-1}x$, we have $\alpha=1,\beta=-1$. And, \begin{align*}
                    \int \frac{\sin x}{\cos x}\D x &= -\int \frac{1}{\cos x}\D\bra{\cos x}\\
                &= -\ln\abs{\cos x} + C.\numberthis
                \end{align*}
                \item[7.] Since $\frac{1}{\sin x}=\sin^{-1}x\cos^{0}x$, we have $\alpha=-1,\beta=0$. From Homework 2 Question 11 Claim 1, we have proven that \[\int \frac1{\sin x}\D x = \ln\abs{\csc x-\cot x} + C. \numberthis\]
                \item[8.] Since $\frac{\cos x}{\sin x}=\sin^{-1}x\cos^{1}x$, we have $\alpha=-1,\beta=1$. And,\begin{align*}
                \int\frac{\cos x}{\sin x}\D x &= \int \frac{1}{\sin x}\D\bra{\sin x}\\
                &= \ln\abs{\sin x} + C.\numberthis
                \end{align*}
                \item[9.] Since $\frac{1}{\sin x\cos x}=\sin^{-1}x\cos^{-1}x$, we have $\alpha=-1,\beta=-1$. And, \begin{align*}
                    \int\frac{1}{\sin x\cos x}\D x &=\int \frac{1}{\frac12\sin 2x}\D x\\
                &=\int\frac{2}{\sin 2x}\cd\frac12\D\bra{2x}\\
                &=\int\frac{1}{\sin 2x}\D\bra{2x}\\
                &=\ln\abs{\csc 2x-\cot 2x} + C.\numberthis
                \end{align*}
        \end{mathnote}
    \end{enumerate}
    \end{question}
    \begin{question}{}{q2}
        By taking the substitution $z=\sin^2x$, show that \[\int \sin ^\alpha x\cos^\beta x\D x=\frac12 J_{\frac{\beta-1}{2},\frac{\alpha-1}{2}},\] where \[J_{p,q}=\int(a+bz)^p z^q\D z, a,b\in\R, p,q\in\Q\] stands for a \textbf{binomial integral} (introduced in the previous homework).
        \qbreak
        \begin{mathnote}
            \begin{center}
                \begin{tikzpicture}[ang/.style={draw,angle eccentricity=1.5, angle radius=0.75cm}]
                    \coordinate (A) at (0,0);
                    \coordinate (B) at (4,0);
                    \coordinate (C) at (4,3);
                    
                    \draw (A) -- node[below]{$\sqrt{1-z^2}$} (B) -- node[right]{$z$} (C) -- node[above left]{$1$} cycle
                    pic ["$x$",ang]{angle=B--A--C};
                \end{tikzpicture}
            \end{center}

            We first let $z=\sin x$, where $\D z = \cos x \D x$, then by Pathagorean Theorem we have \begin{align*}
                \int \sin ^\alpha x\cos^\beta x\D x&=\int z^\alpha \bra{\sqrt{1-z^2}}^{\beta-1}\D z\\
                &=\int z^\alpha \bra{1-z^2}^{\frac{\beta-1}{2}}\D z,\\
                \alt{now let $u=z^2$ where $\D u=2z\D z$, then}
                &=\frac12\int u^{\frac{\alpha-1}{2}}\bra{1-u}^{\frac{\beta-1}{2}}\D u\\&=\frac12\int u^{\frac{\alpha-1}{2}}\bra{1-u}^{\frac{\beta-1}{2}}\D u\\
                &=\frac12\int \bra{1-u}^{\frac{\beta-1}{2}}u^{\frac{\alpha-1}{2}}\D u,\\
                \alt{by letting $a=1$, $b=-1$, $p=\frac{\beta-1}{2}$, $q=\frac{\alpha-1}{2}$, we can see the integral becomes}
                &=\frac12 J_{\frac{\beta-1}{2},\frac{\alpha-1}{2}},\\
                \alt{which is precisely our desired result.}
            \end{align*}
        \end{mathnote}
    \end{question}
    \begin{question}{}{q3}
        Prove that $\forall\alpha,\beta\in\Z, \int \sin^\alpha x\cos^\beta x\D x$ can be reduced into one of the 9 integrals in Question \ref{question:q1}.
        \begin{hint}
            You might want to recall the four iteration formula of binomial integrals, which you have proved by yourself in the previous homework. Alternately you can integrate by parts as well.
        \end{hint}
        \qbreak
        \begin{proof}
            The 4 recursive formula we have proven in Homework 2 are the followings:
            \begin{enumerate}
                \item If $p\neq -1$, then \[\begin{aligned}J_{p,q}=-\frac{(a+bz)^{p+1}z^{q+1}}{a(p+1)}+\frac{p+q+2}{a(p+1)}J_{p+1,q},\end{aligned}\]
                \item If $q\neq -1$, then \[\begin{aligned}J_{p,q}=\frac{(a+bz)^{p+1}z^{q+1}}{a(q+1)}-b\frac{p+q+2}{a(q+1)}J_{p,q+1},\end{aligned}\]
                \item If $p+q\neq-1$, then \[J_{p,q} =\frac{(a+bz)^{p}z^{q+1}}{p+q+1}+\frac{ap}{p+q+1}J_{p-1,q},\]  
                \item If $p+q\neq-1$, then  \[J_{p,q} =\frac{(a+bz)^{p+1}z^{q}}{b(p+q+1)}-\frac{aq}{b(p+q+1)}J_{p,q-1}. \]
            \end{enumerate}

            Now, first we change $p,q$ to $\beta,\alpha$, let $p=\frac{\beta-1}{2}, q=\frac{\alpha-1}{2}$, then $p\neq-1\iff \beta\neq-1$, $q\neq-1\iff\alpha\neq-1$, and $p+1\neq-1\iff \alpha+\beta\neq0$. So, the equations become 
            \begin{enumerate}
                \item If $\beta\neq-1$, then \[J_{\frac{\beta-1}{2},\frac{\alpha-1}{2}} = -\frac{(a+bz)^{\frac{\beta-1}{2}+1}z^{\frac{\alpha-1}{2}+1}}{a(\frac{\beta-1}{2}+1)}+\frac{\frac{\beta-1}{2}+\frac{\alpha-1}{2}+2}{a(\frac{\beta-1}{2}+1)}J_{\frac{\beta-1}{2}+1,\frac{\alpha-1}{2}},\]
                \item If $\alpha\neq-1$, then \[J_{\frac{\beta-1}{2},\frac{\alpha-1}{2}}=\frac{(a+bz)^{\frac{\beta-1}{2}+1}z^{\frac{\alpha-1}{2}+1}}{a(\frac{\alpha-1}{2}+1)}-b\frac{\frac{\beta-1}{2}+\frac{\alpha-1}{2}+2}{a(\frac{\alpha-1}{2}+1)}J_{\frac{\beta-1}{2},\frac{\alpha-1}{2}+1},\]
                \item If $\alpha+\beta\neq0$, then \[J_{\frac{\beta-1}{2},\frac{\alpha-1}{2}}=\frac{(a+bz)^{\frac{\beta-1}{2}}z^{\frac{\alpha-1}{2}+1}}{\frac{\beta-1}{2}+\frac{\alpha-1}{2}+1}+\frac{a\frac{\beta-1}{2}}{\frac{\beta-1}{2}+\frac{\alpha-1}{2}+1}J_{\frac{\beta-1}{2}-1,\frac{\alpha-1}{2}},\]
                \item If $\alpha+\beta\neq0$, then \[J_{\frac{\beta-1}{2},\frac{\alpha-1}{2}}=\frac{(a+bz)^{\frac{\beta-1}{2}+1}z^{\frac{\alpha-1}{2}}}{b\bra{\frac{\beta-1}{2}+\frac{\alpha-1}{2}+1}}-\frac{a\frac{\alpha-1}{2}}{b \bra{\frac{\beta-1}{2}+\frac{\alpha-1}{2}+1}}J_{\frac{\beta-1}{2},\frac{\alpha-1}{2}-1}.\]
            \end{enumerate}

            \newpage

            Now, let $a=1, b=-1$, then the equations become

            \begin{enumerate}
                \item If $\beta\neq-1$, then \[J_{\frac{\beta-1}{2},\frac{\alpha-1}{2}} = -\frac{2(1-z)^{\frac{\beta-1}{2}+1}z^{\frac{\alpha-1}{2}+1}}{\beta+1}+\frac{\beta+\alpha+2}{\beta+1}J_{\frac{(\beta+2)-1}{2},\frac{\alpha-1}{2}},\numberthis\label{first}\]
                \item If $\alpha\neq-1$, then \[J_{\frac{\beta-1}{2},\frac{\alpha-1}{2}}=\frac{2(1-z)^{\frac{\beta-1}{2}+1}z^{\frac{\alpha-1}{2}+1}}{\alpha+1}+\frac{\beta+\alpha+2}{\alpha+1}J_{\frac{\beta-1}{2},\frac{(\alpha+2)-1}{2}},\numberthis\label{second}\]
                \item If $\alpha+\beta\neq0$, then \[J_{\frac{\beta-1}{2},\frac{\alpha-1}{2}}=\frac{2(1-z)^{\frac{\beta-1}{2}}z^{\frac{\alpha-1}{2}+1}}{\beta+\alpha}+\frac{\beta-1}{\beta+\alpha}J_{\frac{(\beta-2)-1}{2},\frac{\alpha-1}{2}},\numberthis\label{third}\]
                \item If $\alpha+\beta\neq0$, then \[J_{\frac{\beta-1}{2},\frac{\alpha-1}{2}}=-\frac{2(1-z)^{\frac{\beta-1}{2}+1}z^{\frac{\alpha-1}{2}}}{\beta+\alpha}+\frac{\alpha-1}{\beta+\alpha}J_{\frac{\beta-1}{2},\frac{(\alpha-2)-1}{2}}.\numberthis\label{fourth}\]
            \end{enumerate}

            With these 4 formulas, we can finally prove that $\forall\alpha,\beta\in\Z, \int \sin^\alpha x\cos^\beta x\D x$ can be reduced into one of the 9 integrals in Question \ref{question:q1}.

            To this end, we consider 4 cases:

            \begin{description}
                \item[Case 1: $\beta<-1$] We apply \eqref{first}, that is, \begin{align*}
                    \int\sin^\alpha x\cos^\beta x\D x&= J_{\frac{\beta-1}{2},\frac{\alpha-1}{2}}\\
                    &=-\frac{2(1-z)^{\frac{\beta-1}{2}+1}z^{\frac{\alpha-1}{2}+1}}{\beta+1}+\frac{\beta+\alpha+2}{\beta+1}J_{\frac{(\beta+2)-1}{2},\frac{\alpha-1}{2}}\\
                    &=-\frac{2(1-z)^{\frac{\beta-1}{2}+1}z^{\frac{\alpha-1}{2}+1}}{\beta+1}+\frac{\beta+\alpha+2}{\beta+1}\int\sin^\alpha x\cos^{\beta+2} x\D x.\\
                \end{align*}
                We repeat this until $\beta\in\{-1,0,1\}$. Since $\beta<-1\implies\beta+2<1$, and $\beta<\beta+2$, thus this process always terminate and is valid.
                \item[Case 2: $\alpha<-1$]  We apply \eqref{second}, that is, \begin{align*}
                    \int\sin^\alpha x\cos^\beta x\D x &= J_{\frac{\beta-1}{2},\frac{\alpha-1}{2}}\\
                    &=\frac{2(1-z)^{\frac{\beta-1}{2}+1}z^{\frac{\alpha-1}{2}+1}}{\alpha+1}+\frac{\beta+\alpha+2}{\alpha+1}J_{\frac{\beta-1}{2},\frac{(\alpha+2)-1}{2}}\\
                    &=\frac{2(1-z)^{\frac{\beta-1}{2}+1}z^{\frac{\alpha-1}{2}+1}}{\alpha+1}+\frac{\beta+\alpha+2}{\alpha+1}\int\sin^{\alpha+2} x\cos^{\beta} x\D x.\\
                \end{align*}
                Again, we repeat this until $\alpha\in\{-1,0,1\}$. Since $\alpha<-1\implies\alpha+2<1$, and $\alpha<\alpha+2$, thus this process always terminate and is valid.
                \item[Case 3: $\beta>1$] Becasue of our Case 2, we may assume $\alpha\geq-1$. Thus, in this case we have $\alpha+\beta>0$, apply \eqref{third} and we have \begin{align*}
                    \int\sin^\alpha x\cos^\beta x\D x &= J_{\frac{\beta-1}{2},\frac{\alpha-1}{2}}\\
                    &=\frac{2(1-z)^{\frac{\beta-1}{2}}z^{\frac{\alpha-1}{2}+1}}{\beta+\alpha}+\frac{\beta-1}{\beta+\alpha}J_{\frac{(\beta-2)-1}{2},\frac{\alpha-1}{2}}\\
                    &=\frac{2(1-z)^{\frac{\beta-1}{2}}z^{\frac{\alpha-1}{2}+1}}{\beta+\alpha}+\frac{\beta-1}{\beta+\alpha}\int\sin^\alpha x\cos^{\beta-2} x\D x.\\
                \end{align*}
                Again, we repeat this until $\beta\in\{-1,0,1\}$. Since $\beta>1\implies\beta-2>-1$, and $\beta>\beta-2$, thus this process always terminate and is valid.
                \item[Case 4: $\alpha>1$] Becasue of our Case 1, we may assume $\beta\geq-1$. Thus, in this case we have $\alpha+\beta>0$, apply \eqref{fourth} and we have \begin{align*}
                    \int\sin^\alpha x\cos^\beta x\D x &= J_{\frac{\beta-1}{2},\frac{\alpha-1}{2}}\\
                    &=-\frac{2(1-z)^{\frac{\beta-1}{2}+1}z^{\frac{\alpha-1}{2}}}{\beta+\alpha}+\frac{\alpha-1}{\beta+\alpha}J_{\frac{\beta-1}{2},\frac{(\alpha-2)-1}{2}}\\
                    &=-\frac{2(1-z)^{\frac{\beta-1}{2}+1}z^{\frac{\alpha-1}{2}}}{\beta+\alpha}+\frac{\alpha-1}{\beta+\alpha}\int\sin^{\alpha-2} x\cos^{\beta} x\D x.\\
                \end{align*}
                Again, we repeat this until $\alpha\in\{-1,0,1\}$. Since $\alpha>1\implies\alpha-2>-1$, and $\alpha>\alpha-2$, thus this process always terminate and is valid.
            \end{description}

            Now, for all $\alpha,\beta\in\Z$, by the resurcive formulas, we can always reduce the integral into $\int \sin^\alpha x\cos^\beta x\D x$ where $\alpha,\beta\in\{-1,0,1\}$, where all $3\times3=9$ possible cases are covered by our Question \ref{question:q1}. Therefore, we have shown that $\forall\alpha,\beta\in\Z, \int \sin^\alpha x\cos^\beta x\D x$ can be reduced into one of the 9 integrals in Question \ref{question:q1}, as needed, this completes our proof.

        \end{proof}
    \end{question}
    \begin{question}{}{q4}
        From there, conclude that $\sin^\alpha x\cos ^\beta x$ is integrable in finite terms $\forall \alpha,\beta\in\Z.$
        \qbreak
        \begin{mathnote}
            This conclusion directly comes from our proof of Question \ref{question:q3}. First, the 9 base cases can be integrated in finite terms, as shown in Question \ref{question:q1}. Then, by the recursive formulas in Question \ref{question:q3}, we can always reduce the integral into one of the 9 base cases, combining with the intermediate terms, we have the integral is integrable in finite terms $\forall \alpha,\beta\in\Z.$
            
            \hfill
        \end{mathnote}
    \end{question}
\end{exercise}
\newpage
\resetcounter{question}
\begin{exercise}{}{}
    \begin{note}
        This exercise aims at demonstrating that various approaches can be used to ocmpute the same integral. More precisely, compute the indefinite integral \[\int \frac{\D x}{\sqrt{a^2+x^2}}, a>0.\]
    \end{note}
    \begin{question}{}{e2q1}
        Using trigonometric substitution $x=a\tan t,  -\frac\pi2<t<\frac\pi2$.
        \qbreak
        \begin{mathnote}
            Let $x=a\tan t$, then isolate $t$ we have $t=\arctan\bra{\frac{x}{a}}$ and $\D t=\frac1a\cd\frac{1}{1+\bra{\frac{x^2}{a^2}}}\D x=\frac{a}{x^2+a^2}\D x$. 
            \begin{center}
                \begin{tikzpicture}[ang/.style={draw,angle eccentricity=1.5, angle radius=0.75cm}]
                    \coordinate (A) at (0,0);
                    \coordinate (B) at (2,0);
                    \coordinate (C) at (2,1.5);
                    
                    \draw (A) -- node[below]{$a$} (B) -- node[right]{$x$} (C) -- node[above left]{$\sqrt{x^2+a^2}$} cycle
                    pic ["$t$",ang]{angle=B--A--C};
                \end{tikzpicture}
            \end{center}Now, we substitute $t$ into $x$ and we have \begin{align*}
                \int\frac{\D x}{\sqrt{a^2+x^2}} &= \int \frac{1}{\sqrt{a^2+a^2\tan^2 t}}\cd\frac{x^2+a^2}{a}\D t\\
                &=\int \frac{1}{|a|}\frac{1}{\sqrt{1+\tan^2 t}}\frac{a^2\tan^2t+a^2}{a}\D t\\
                &=\int\frac{\tan^2 t + 1}{\sqrt{\tan^2 t + 1}}\D t\\
                &=\int\frac{\sec^2 t}{\sqrt{\sec^2 t}}\D t,\\
            \alt{since $t\in\bra{-\frac\pi2,\frac\pi2}$, this gives $\sec t>0$, thus }
            &= \int \sec t\D t\\
            &= \int \frac{1}{\cos t}\D t,\\
            \alt{from \eqref{eq3} we have}
            &= \ln\abs{\sec t + \tan t}+C\\
            &= \ln\abs{\frac{\sqrt{x^2+a^2}}{a}+\frac{x}{a}} + C\\
            &= \ln\abs{\sqrt{x^2+a^2}+x} - \ln\abs{a} + C\\
            &= \ln\abs{x+\sqrt{x^2+a^2}} + C.\\
            \end{align*}
        \end{mathnote}
    \end{question}
    \begin{question}{}{e2q2}
        Using hyperbolic substitution $x=a\sinh t, t\in\R$.
        \qbreak
        \begin{mathnote}
            Let $x=a\sinh t$, then $\D x=a\cosh t\D t$ and $t=\operatorname{arcsinh}\bra{\frac{x}{a}}$, substitute into the integral we have \begin{align*}
                \int \frac{1}{\sqrt{a^2+x^2}}\D x &= \int \frac{1}{\sqrt{a^2+a^2\sinh^2 t}}\cd a\cosh t\D t\\
                &=\int \frac{a}{|a|}\frac{\cosh t}{\sqrt{1+\sinh t}} \D t\\
                &=\int \frac{\cosh t}{\sqrt{\cosh^2 t}} \D t\\
                &=\int \frac{\cosh t}{\cosh t} \D t\\
                &=\int 1\D t\\
                &=t+C\\ 
                &=\operatorname{arcsinh}\bra{\frac{x}{a}}+C\\
                &=\ln\abs{\frac{x}{a}+\sqrt{\frac{x^2}{a^2}+1}}+C\\
                &=\ln\abs{\frac{x+\sqrt{x^2+a^2}}{a}}+C\\
                &=\ln\abs{x+\sqrt{x^2+a^2}}-\ln\abs{a}+C\\
                &=\ln\abs{x+\sqrt{x^2+a^2}}+C.\\
            \end{align*}
        \end{mathnote}
    \end{question}
    \begin{question}{}{e2q3}
        Using typer I Euler substitution $\sqrt{x^2+a^2}=x+t$.
        \qbreak
        \begin{mathnote}
            Let $\sqrt{x^2+a^2}=x+t$, square both sides we have $x^2+a^2=(x+t)^2=x^2+2xt+t^2$, cancel $x^2$ the equation becomes $2xt=a^2-t^2$, divide both sides we get $x=\frac{a^2-t^2}{2t}$, then differentiate both sides, the equation now changes to $\D x = \frac{-4t^2\D t - 2(a^2-t^2)\D t}{4t^2}=-\frac{a^2+t^2}{2t^2}\D t$. Substitute everything back into the integral we have \begin{align*}
                \int \frac{\D x}{\sqrt{a^2+x^2}} &= -\int \frac{1}{\frac{a^2-t^2}{2t}+t}\cd\frac{a^2+t^2}{2t^2}\D t\\
                &=-\int \frac{2t}{a^2-t^2+2t^2}\cd\frac{a^2+t^2}{2t^2}\D t\\
                &=-\int \frac{a^2+t^2}{a^2+t^2}\frac{1}{t}\D t\\
                &=-\ln\abs{t}+C\\
                &=-\ln\abs{\sqrt{x^2+a^2}-x}+C\\
                &=\ln\abs{\frac1{\sqrt{x^2+a^2}-x}\cd\frac{\sqrt{x^2+a^2}+x}{\sqrt{x^2+a^2}+x}}+C\\
                &=\ln\abs{\frac{\sqrt{x^2+a^2}+x}{\sqrt{x^2+a^2}-x^2}}+C\\
                &=\ln\abs{\frac{\sqrt{x^2+a^2}+x}{a^2}}+C\\
                &=\ln\abs{x+\sqrt{x^2+a^2}}-\ln\abs{a}+C\\
                &=\ln\abs{x+\sqrt{x^2+a^2}}+C.\\
            \end{align*}
        \end{mathnote}
    \end{question}

    \begin{question}{}{e2q4}
        Using type II Euler substitution $\sqrt{x^2+a^2}=xt+a$.
        \qbreak
        \begin{claim}{}{cl1}
            \[\int\frac{1}{1-x^2}\D x=\frac12\ln\abs{\frac{1+x}{1-x}}\D x.\]
            \begin{proof}[Proof of Claim \ref{claim:cl1}]
                We will use our claim from Homework 2:
                \[\int \frac{1}{\sin x}\D x = \ln\abs{\csc x - \cot x} + C.\]

                First, let $x=\cos t$ where $\D x=-\sin t\D t$. Then, substitute $t$ into our integral we get \begin{align*}
                    \int \frac{1}{1-x^2}\D x &= -\int\frac{1}{1-\cos ^2 t}\cd\sin t\D t\\
                    &= -\int\frac{\sin t}{\sin ^2 t}\D t\\
                    &= -\int\frac{1}{\sin t}\D t\\
                    &= -\ln\abs{\csc t - \cot t} + C.\\
                    \alt{Now, consider the Pythagorean Theorem and the triangle,}
                    &\begin{tikzpicture}[ang/.style={draw,angle eccentricity=1.5, angle radius=0.75cm}]\coordinate (A) at (0,0);\coordinate (B) at (2,0);\coordinate (C) at (2,1.5);\draw (A) -- node[below]{$1$} (B) -- node[right]{$\sqrt{1-x^2}$} (C) -- node[above left]{$x$} cycle pic ["$t$",ang]{angle=B--A--C}; \end{tikzpicture}
                    \alt{this gives}
                    &= -\ln\abs{\frac{1}{\sqrt{1-x^2}} - \frac{x}{\sqrt{1-x^2}}} + C\\
                    &= \ln\abs{\frac{\sqrt{1-x^2}}{1-x}} + C\\
                    &= \ln\abs{\bra{\frac{1-x^2}{(1-x)^2}}^{\frac12}} + C\\
                    &= \frac12\ln\abs{\frac{(1-x)(1+x)}{(1-x)^2}} + C\\
                    &= \frac12\ln\abs{\frac{1+x}{1-x}} + C.
                \end{align*}
                
            \end{proof}
        \end{claim}
        \begin{mathnote}
            \begin{align*}
                \alt{Let $t=\frac{\sqrt{x^2+a^2}-a}{x}$, then}
                \sqrt{x^2+a^2}&=xt+a,\\
                \alt{square both sides,}
                x^2+a^2&=x^2t^2+2xta+a^2,\\
                \alt{cancel $a^2$ and divide by $x$,}
                x&=xt^2+2at,\\
                \alt{isolate $x$ we get}
                x&=\frac{2at}{1-t^2}\\
                \D x&=2a\frac{1-t^2-t(-2t)}{(1-t^2)^2}\D t\\
                &=2a\frac{1+t^2}{(1-t^2)^2}\D t.\\
                \alt{Substitute $t$ into the integral,}
                \int \frac{\D x}{\sqrt{a^2+x^2}}&=\int \frac{1}{\bra{\frac{2at}{1-t^2}}t+a}\cd\frac{2a(1+t^2)}{(1-t^2)^2}\D t\\
                &=\int\frac{1-t^2}{2at^2+a(1-t^2)}\cd\frac{2a(1+t^2)}{(1-t^2)^2}\D t\\
                &=\int\frac{2(1+t^2)}{2t^2+1-t^2}\cd\frac{1}{1-t^2}\D t\\
                &=2\int\frac{1+t^2}{1+t^2}\cd\frac{1}{1-t^2}\D t\\
                &=2\int\frac{1}{1-t^2}\D t,\\
                \alt{by Claim \ref{claim:cl1} we have}
                &=\ln\abs{\frac{1+t}{1-t}}+C,\\
                \alt{now substitute $x$ back into the integral,}
                &=\ln\abs{\frac{1+\frac{\sqrt{x^2+a^2}-a}{x}}{1-\frac{\sqrt{x^2+a^2}-a}{x}}}+C\\
                &=\ln\abs{\frac{x+\sqrt{x^2+a^2}-a}{x-\sqrt{x^2+a^2}-a}}+C\\
                &=\ln\abs{\frac{(x-a)+\sqrt{x^2+a^2}}{(x-a)-\sqrt{x^2+a^2}}\cd\frac{(x-a)+\sqrt{x^2+a^2}}{(x-a)+\sqrt{x^2+a^2}}}+C\\
                &=\ln\abs{\frac{\bra{(x-a)+\sqrt{x^2+a^2}}^2}{(x-a)^2-x^2-a^2}}+C\\
                &=\ln\abs{\frac{\bra{(x-a)+\sqrt{x^2+a^2}}^2}{x^2-2ax+a^2-x^2-a^2}}+C\\
                &=\ln\abs{\frac{\bra{(x-a)+\sqrt{x^2+a^2}}^2}{-2ax}}+C\\
                &=\ln\abs{\frac{x^2+x^2+x\sqrt{x^2+a^2}+x\sqrt{x^2+a^2}}{2ax}\right.+\\
                &\quad\quad\quad\left.\frac{-ax+ax+a^2-a^2-a\sqrt{x^2+a^2}+a\sqrt{x^2+a^2}}{2ax}}+C\\
                &=\ln\abs{\frac{2x^2+2x\sqrt{x^2+a^2}}{2ax}}+C\\
                &=\ln\abs{\frac{x+\sqrt{x^2+a^2}}{a}}+C\\
                &=\ln\abs{x+\sqrt{x^2+a^2}}-\ln\abs{a}+C\\
                &=\ln\abs{x+\sqrt{x^2+a^2}}+C.\\
            \end{align*}
        \end{mathnote}
    \end{question}
\end{exercise}

\end{document}