\documentclass[12pt, sepia, brown, 0.5in]{alittlebear}

\def\name{Joseph Siu}
\def\course{MAT159: Analysis II}
\def\headername{Homework }
\def\headernum{5}

\usepackage{comment}

\begin{document} 

\newn{
    I, Joseph Siu, affirm that this assignment represents entirely my own efforts. I confirm that:
    \begin{itemize}
        \item I have not copied any portion of this work.
        \item I have not allowed someone else in this course to copy this work.
        \item This is the final version of my assignment and not a draft.
        \item I understand the consequences of violating the University's academic integrity policies as outlined in the \textit{Code of Behavior on Academic Matters}.
    \end{itemize}
}


\newt{1}{
    Recall that we have proved the following theorem in class:

    Let $f:[a,b]\to\R$ be a bounded function. Then the following statements are equivalent:
    \begin{enumerate}
        \item $f\in\mathfrak{R}[a,b]$.
        \item (Darboux Criterion) $\overline{I}(f)=\underline{I}(f)$.
        \item $\forall \ep >0,\forall \delta >0, \exists \Gamma \in\Omega_{[a,b]}$ s.t. \[\sum_{\substack{0\leq k\leq n-1 \\ V_f(I_k)>\ep}}\abs{I_k}<\delta.\]
        \item (Du Bois Raymond Criterion) $\forall \ep > 0, \forall\de>0,\exists n\in \N$ and $(a_1,b_1), (a_2,b_2),\ldots,(a_n,b_n),$ such that \[\bra{D_{f}(\ep,[a,b])\subseteq\bigcup_{1\leq k\leq n}(a_k,b_k)}\land\bra{\sum_{k=1}^{n}(b_k-a_k)<\de}.\]
        \item (Lebesgue Criterion) $f$ is continuous almost everywhere on $[a,b]$.
    \end{enumerate}
}
\begin{comment}
    By definition of limit $\forall\xi>0,\exists\ga>0$ such that $0<|\de|<\ga\implies|V_f(I_\de(x))-\eta|<\xi$. By specializing $\xi=\ep$, and fix such $\ga$, we have $0<|\de|<\ga\implies ||V_f(I_\de(x))|-|\eta||<\ep$. Since for all $0<\delta'<\delta$ we have $I_\delta(x)\supseteq I_{\delta'}(x)$, which implies $V_f(I_\delta(x))\geq V_f(I_{\delta'}(x))$. Thus, this implies $\eta< V_f(I_\de(x))$. Thus, as all our $\eta, V_f(I_\de(x)),\ep$ are positive here, we get $V_f(I_\de(x))-\eta>\ep$ and $V_f(I_\de(x))>\ep+\eta$.

    Now, fix an open neighborhood $I_\zeta(x)\subseteq[c,d]$ of $x$, redefine $\ga:=\min\{\ga, \frac{\zeta}2\}$, then we have $V_f([c,d])\geq V_f(I_\zeta(x))\geq V_f(I_\de(x))>\ep+\eta>\ep$ for all $0<\de<\ga\leq\zeta$, which completes our proof.
\end{comment}



\newq{1}{
    Prove that $5\Ra4\Ra3\Ra2\Ra1\Ra5$. In each part, you should not use criterion other than the two involved in the statement.
    \qbreak

    \newl{1}{
        Fix $\ep>0$. If $x\in D_f(\ep,[a,b])$, then $V_f([c,d])>\ep$ for any $[c,d]\subseteq[a,b]$ containing $x$. 

        \qbreak

        \newp{
            If $x\in D_f(\ep,[a,b])$, by definition we have $V_f(x)>\ep$, that is, $\eta:=\lim_{\de\to0}V_f(I_\delta(x))>\ep$. Fix an open neighborhood $I_\zeta(x)\subseteq[c,d]$ of $x$. Since for all $0<\delta'<\delta$ we have $I_\delta(x)\supseteq I_{\delta'}(x)$, which implies $V_f(I_\delta(x))\geq V_f(I_{\delta'}(x))$. Thus, since all sequences of decreasing $\de$ are monotonely decreasing, we have that $\eta< V_f(I_\de(x))$ for all $\de>0$. Thus, $V_f(I_\zeta(x))\geq \eta >\ep$, this implies our proof.
        }
    }

    \newl{4}{
        If $\forall x\in [a,b], V_f(x)\leq\frac{\ep}{3}$, then for all positive epsilon there exists a finite partition $\Ga$ of $[a,b]$ such that $V_f(I_i)\leq\ep$ for all partitioned intervals $I_i$ of $\Ga$.

        \qbreak

        \newp{
            $V_f(x)\leq\frac{\ep}{3}$ implies there exists an open neighborhood $I_\de(x)$ of $x$ such that $V_f(I_\de(x))\leq\frac{\ep}{3}$. Then, we can see $\{I_\de(x)\}_{x\in[a,b]}$ forms an open cover of $[a,b]$. Moreover, since $[a,b]$ is closed and bounded thus compact, by Borel-Lebesgue / Heine-Borel there exists a finite subcover $\{J_i\}_{1\leq i\leq n}$ of $\{I_\de(x)\}_{x\in[a,b]}$. Then, construct a partition $\Ga$ based on the endpoints of the intervals in the finite subcover $\{J_i\}_{1\leq i\leq n}$.

            Now, for any partitioned interval $[\al,\be]$ of $\Ga$, we have $(\al,\be)\subseteq J_i$ for some $1\leq i\leq n$. Since $V_f(I_\de(\al))\leq\frac{\ep}{3}$, $V_f(J_i)\leq\frac{\ep}{3}$, and $V_f(I_\de(\be))\leq\frac{\ep}{3}$, we have $V_f([\al,\be])\leq \ds\sup_{x,y\in I_\de(\al)\cup J_i\cup I_\de(\be)}|f(x)-f(y)|\leq\frac{\ep}{3}+\frac{\ep}{3}+\frac{\ep}{3}=\ep$.

            Since we have consturcted such finite partition $\Ga$ for arbitrary $\ep>0$, this completes our proof.
        }
    }
}

\newp{$(5\Ra4)$

    Assume $f$ is continuous almost everywhere on $[a,b]$. By definition of almost everywhere, this implies the set of discontinuous points of $f$ forms a null set $D$.

    By definition of continuity, we have $$\forall \ep>0, \exists \de>0, \forall x\in I_\de(x), |f(x)-f(x_0)|<\ep.$$ Fix $\ep = \frac{\xi}{4}$, then $V_f(I_\de(x))\leq\frac{\xi}{2}$. 

    For continuous points $x$ in $[a,b]$, we construct the open neighborhood $I_\de(x)$ as above, let $I_x$ denote such open neighborhood of $x$. For discontinuous points $x'$ in $[a,b]$, since $x'\in D$ and $D$ forms a null set, by definition this implies \[\forall \de >0, \exists\{I_i\}_{i\in\N}, \bra{D\subseteq\bigcup_{i\in\N}I_i}\land\bra{\sum_{i=1}^\infty\abs{I_i}<\de}.\] Fix $\de=\frac{\xi}{4}$, then we have $\exists\{I_i\}_{i\in\N}, \bra{D\subseteq\bigcup_{i\in\N}I_i}\land\bra{\sum_{i=1}^\infty\abs{I_i}<\frac{\xi}{4}}$, fix such $\{I_i\}_{i\in\N}$. Now, let $I_{x'}$ be the open interval that covers $x'$. 
    
    So, since $[a,b]$ is closed and bounded that compact, and $\{I_x\}_{x\in [a,b]\setminus D}\cup\{I_{x'}\}_{x'\in D}$ forms a cover of $[a,b]$, hence by Borel-Lebesgue / Heine-Borel Theorem this implies a finite subcover $\{J_i\}_{1\leq i\leq N}$ of $\{I_x\}_{x\in [a,b]\setminus D}\cup\{I_{x'}\}_{x'\in D}$ where $N$ is the number of intervals in the finite subcover. 

    Now, let $\Ga$ be the partition based on the endpoints of the intervals in the finite subcover $\{J_i\}_{1\leq i\leq N}$, consider 2 of the cases of the partitioned intervals $[\al,\be]$: 

    \begin{proofcases}
        \case $(\al,\be)\subseteq I_x$ for some $x\in [a,b]\setminus D$. 

        \indenv{
            In this case we have $V_f([\al,\be])\leq V_f(I_x)\leq\frac{\xi}{2}<\xi$. That is, $D_f(\xi,[a,b])\cap [\al,\be]=\nil$. We may ignore these intervals for proving Du Bois Raymond Criterion (4). 
        }

        \case $(\al,\be)\subseteq I_{x'}$ for some $x'\in D$.

        \indenv{
            In this case we have \[\bra{[\al,\be]\subseteq I_{x'}}\land\bra{|I_{x'}|<\frac{\xi}{4}}\implies (\beta-\al)<\frac{\xi}{4}.\] 
                        
            Let $M$ denote the number of all such $[\al,\be]$, since the cover $\{J_i\}_{1\leq i\leq N}$ is finite, this implies the partition of $\Ga$ is finite, thus $M$ also needs to be finite.

            Let $[\al_i,\be_i]$ denote the $i^{\T{th}}$ such interval where $1\leq i\leq M$, this is allowed because of the order of the partition of $\Ga$.
        }
          
    \end{proofcases}

    Since there are only finitely many such $\al$ and $\be$ by our partition $\Ga$, let $N$ be the set containing all partitions of $\Ga$, \ie, containing $a,b$, and all $\al,\be$. Then, since $N$ is a collection of finitely many points, we can see $N$ is also a null set. Hence, by definition of null set, we have \[\forall \de >0, \exists\{I_i\}_{i\in\N}, \bra{N\subseteq\bigcup_{i\in\N}I_i}\land\bra{\sum_{i=1}^\infty\abs{I_i}<\de}.\] Fix $\de=\frac{\xi}{4}$, let $\{K_i\}_{i\in\N}$ be the open cover that covers $N$ and $\sum_{i=1}^\infty |K_i|<\frac{\xi}{4}$. Moreover, since $N$ is finite and closed thus bounded. By Borel-Lebesgue / Heine-Borel Theorem, we can find a finite subcover $\{K'_i\}_{1\leq i\leq P}$ of $\{K_i\}_{i\in\N}$ where $P$ is the number of intervals in the finite subcover.
    
    Now, combining the above 2 cases we have shown that \[\bra{D_f(\xi,[a,b])\subseteq\bra{\bigcup_{1\leq i\leq M}(\al_i,\be_i)}\bigcup\bra{\bigcup_{1\leq i\leq P}K'_i}}\land\bra{\sum_{i=1}^M(\be_i-\al_i)+\sum_{i=1}^P\abs{K'_i}<\frac{\xi}{4}+\frac{\xi}{4}<\xi}.\] 
    
    Hence, since our $\xi$ is arbitrary, for arbitrary $\ep,\de>0$ by letting $\xi:=\min\{\ep,\de\}, n:=M+P$, leting $(a_1,b_1),\ldots,(a_{n},b_{n})$ be the intervals that covers $D_f(\xi,[a,b])$, we have shown that \[\bra{D_f(\ep,[a,b])\subseteq D_f(\xi,[a,b])\subseteq \bigcup_{1\leq k\leq n}(a_k,b_k)}\land\bra{\sum_{k=1}^n(b_k-a_k)<\xi\leq\de}.\] Therefore, $\forall \ep >0, \forall\de>0,\exists n\in \N$ and $(a_1,b_1), (a_2,b_2),\ldots,(a_n,b_n),$ such that \[\bra{D_{f}(\ep,[a,b])\subseteq\bigcup_{1\leq k\leq n}(a_k,b_k)}\land\bra{\sum_{k=1}^{n}(b_k-a_k)<\de},\] this completes our proof. 
}

\newp{$(4\Ra3)$

    Fix $\frac{\ep}{3}>0$, $\de>0$. By Criterion 4 there exist a natural number $n$ and a finite open cover $\{J_k\}_{1\leq k\leq n}$ such that $$\bra{D_f(\frac{\ep}{3},[a,b])\subseteq\bigcup_{1\leq k\leq n}J_k}\land\bra{\sum_{k=1}^n |J_k|<\de}.$$ 

    Fix such $n\in\N$, then there exists $n_1\in\N$ and a finite open cover $\{J_k'\}_{1\leq k\leq n_1}$ such that $$\bra{D_f(\frac{\ep}{3},[a,b])\subseteq\bigcup_{1\leq k\leq n_1}J_k'}\land\bra{\sum_{k=1}^{n_1}|J_k'|<\frac{\de}{n}}.$$

    Let $\Ga$ be the partition based on the endpoints of $\{J_k'\}_{1\leq k\leq n}$. Split the partitioned intervals $[\al,\be]$ into 2 parts:

    \begin{proofcases}
        \case $[\al,\be]\cap D_f(\frac{\ep}{3},[a,b])=\nil$.
        \indenv{
            By Lemma 2, this implies there exists a finite partition of $[\al,\be]$ such that for all partitioned interval $[\gamma,\zeta]$ we have $V_f([\gamma,\zeta])\leq\ep$. Refine our $\Ga$ to include these partitions, let $\Ga^*$ denote the refined partition. 

            Let $m_1$ denote the total number of all such intervals' partitioned intervals. Let $[\al_{k},\be_{k}]$ denote the endpoints of each interval where $0\leq k\leq m_1-1$. After we refine our original $\Ga$ to include the finite partitions for all the ``$[\al,\be]$'', we can see $\ds\sum_{\substack{0\leq k\leq m_1-1 \\ V_f([\al_k,\be_k])>\ep}}|\beta_k-\al_k|=0$ since any partitioned interval $[\al_k,\be_k]$ has the property that $V_f([\al_k,\be_k])\leq\ep$.
        }
        \case $[\al',\be']\cap D_f(\frac{\ep}{3},[a,b])\neq\nil$, that is, $(\al',\be')\subseteq J_i\cup J_j\cup J_k$ for some $1\leq i\leq n, 1\leq j\leq n, 1\leq k\leq n$ by our construction of partitions.
        \indenv{
            Since there can only be $n$ such intervals. By assumption the total length of all such intervals is less than $\delta$. Namely, let $m_2$ denote the number of all such intervals of $\Ga$ (clearly $m_2\leq n$), then $\ds\sum_{1\leq k\leq m_2}|\be_k'-\al_k'|\leq m_2\cd\ds\sum_{k=1}^{n_1}|J_k'|<n\cd\frac{\de}{n}=\de$. 
        }
    \end{proofcases}

    Since these 2 cases cover the entire interval, we conclude \[\ds\sum_{\substack{0\leq k\leq n-1 \\ V_f(I_k)>\ep}}|I_k|\leq0+\ds\sum_{\substack{0\leq k\leq m_2-1 \\ V_f(\al_k',\be_k'])>\ep}}|\be_k'-\al_k'|<\de.\] Since $\ep$ and $\de$ are arbitrary, and we have constructed such $\Ga$, this completes our proof.
}

\newp{$(3\Ra 2)$

    For all $\ep>0$ and $\de>0$, by Criterion 3, there exists a partition $\Ga$ such that $$\sum_{\substack{0\leq k\leq n-1 \\ V_f(I_k)>\de}}\abs{I_k}<\frac{\ep}{8|M|+1},$$ where $f$ is bounded by $M$ (integrable implies boundedness).
    
    Fix $\de=\frac{\ep}{4(b-a)+1}$.
    
    Then, by the supremum and infimum definition we have \begin{align*}
        \overline{I}(f)&\leq\sum_{k=0}^{n-1}M_{x_k,x_{k+1}}\Delta x_k\\
        \sum_{k=0}^{n-1}m_{x_k,x_{k+1}}\Delta x_k &\leq \underline{I}(f)\\
        \overline{I}(f)-\underline{I}(f)&\leq\sum_{k=0}^{n-1}M_{x_k,x_{k+1}}\Delta x_k-\sum_{k=0}^{n-1}m_{x_k,x_{k+1}}\Delta x_k\\
        &\leq\sum_{k=0}^{n-1}(M_{x_k,x_{k+1}}-m_{x_k,x_{k+1}})\Delta x_k\\
        &\leq\sum_{k=0}^{n-1}(V_f([x_k,x_{k+1}]))\Delta x_k,\\
        \alt{now we seperate the intervals into $V_f(I_k)>\de$ and $V_f(I_k)\leq\de$ where $I_k=[x_k,x_{k+1}]$, then:}
        &\leq\sum_{\substack{0\leq k\leq n-1 \\ V_f(I_k)>\de}}(V_f(I_k))\Delta x_k+\sum_{\substack{0\leq k\leq n-1 \\ V_f(I_k)\leq\de}}(V_f(I_k))\Delta x_k,\\
        \alt{since $f$ is bounded by $M$, by our assumption then we have:}
        &\leq2M\sum_{\substack{0\leq k\leq n-1 \\ V_f(I_k)>\de}}\Delta x_k+\sum_{\substack{0\leq k\leq n-1 \\ V_f(I_k)\leq\de}}(V_f(I_k))\Delta x_k\\
        &\leq\frac{\ep}{4} + \sum_{\substack{0\leq k\leq n-1 \\ V_f(I_k)\leq\de}}(V_f(I_k))|I_k|,\\
        &\leq\frac{\ep}{4}+ \frac{\ep}{4(b-a)+1}\sum_{\substack{0\leq k\leq n-1 \\ V_f(I_k)\leq\de}}|I_k| \\
        &\leq \frac{\ep}{4} + \frac{\ep}{4}\\
        &\leq\frac{\ep}{2}\\
        &<\ep
    \end{align*}
    Since $\ep>0$ is arbitrary, we conclude $\overline{I}(f)-\underline{I}(f)=0$, therefore $\overline{I}(f)=\underline{I}(f)$ as needed.
}

\newp{$(2\Ra1)$

    By Proposition 2.3 we have for any two partitions $\Ga_1,\Ga_2\in\Omega[a,b]$, $\underline{S}(f,\Ga_1)\leq \overline{S}(f,\Ga_2).$ So, by definition for any marked partition $(\Ga,\eta)\in\Omega^*[a,b]$ we have \[\underline{S}(f,\Ga_1)\leq\sum_{i=0}^{n-1}f(\eta_i)\Delta x_i\leq \overline{S}(f,\Ga_2).\] By letting $\norm{\Gamma}\to0$, we have \[\underline{I}(f)\leq\sum_{i=0}^{n-1}f(\eta_i)\Delta x_i\leq\overline{I}(f).\] Since $\overline{I}(f)=\underline{I}(f)$, we can claim that $f\in\mathfrak{R}[a,b]$, moreover $\ds\int_a^bf(x)\D x=\overline{I}(f)=\underline{I}(f)$:

    When $\norm{\Gamma}<\de$, this is equivalent to $\norm{\Ga}\to0$, and by squeeze theorem we have that $0-\ds\frac\ep2\leq\ds\sum_{i=0}^{n-1}f(\eta_i)\Delta x_i-\ds\int_a^bf(x)\D x\leq0+\frac\ep2$, which gives $\abs{\ds\sum_{i=0}^{n-1}f(\eta_i)\Delta x_i-\ds\int_a^bf(x)\D x}\leq\frac\ep2<\ep$, since $(\Gamma,\eta)$ is arbitrary, this gives the definition of Riemann integrability, as needed.
}

\newp{$(1\Ra5)$

    We will prove the contrapositive. Assume the discontinuous points of $f$ do not form a null set $N$, namely $$\exists\ep_1>0,\forall \{J_i\}_{i\in\N}\T{ we have }\bra{N\not\subseteq\bigcup_{i\in\N}J_i}\lor\bra{\sum_{i=1}^\infty|J_i|\geq\ep_1}.$$

    Fix such $\ep_1>0$. Let $\de>0$ be arbitrary, let $(\Ga,\eta)$ be arbitrary such that $\norm{\Ga}\leq\de$, let $M$ denotes the number of partitioned intervals.

    Let $\{I_k\}_{k\in M}$ be the partitioned intervals of $\Ga$. Let $a = \min\{V_f(I_k)\}_{k\in M}\geq0$, here $a\neq0$ because of the non-empty discontinuous set $N$. then, consider $\ep := a(\ep_1)>0$. 

    By specialization, and ignore $a,b$ if they are discontinuous, we may consturct an open cover $\{J_i\}_{i\in\N}$ of the discontinuous set $N\setminus\{a,b\}$ such that $\sum_{i=1}^\infty|J_i|\geq\ep_1$ and does not cover $a,b$, since $N$ do not form a null set, so is $N\setminus\{a,b\}$.
    
    Since $\sum_{i=0}^{M-1}|I_i|\geq \sum_{i=1}^\infty |J_i|$ (becasue of the forall quantifier, we may assume the union of the cover $\{J_i\}_{i\in\N}$ is contained within $[a,b]$), we can see that

    \begin{align*}
        \sum_{i=0}^{M-1}V_f(\ep, I_i)|I_i|&\geq a\sum_{i=0}^{M-1}|I_i|\\
        &\geq a\sum_{i=1}^\infty|J_i|\\
        &\geq a\ep_1\geq\ep
    \end{align*}    

    So, by the negation of the Riemann integrability in terms of Aggregated Oscillation, we have shown that $$\exists \ep > 0, \forall \de>0, \exists(\Ga,\eta)\in\Omega^*_{[a,b]},\norm{\Ga}\leq\de\land \sum_{i=0}^{n-1}V_f(I_i)|I_i|\geq \ep,$$ hence the contrapositive is true, which implies $(1\Ra5)$ as needed.
}


\newpage
\newq{2}{
    Using Lebesgue Criterion to study the Riemann integrability for the five examples in assignment 4.
    \qbreak
    \newm{
        \begin{enumerate}
            \item Since $D(x)$ is continuous nowhere (as proven in MAT157 Homework), then the sum of any open cover covering $[0,1]$ must be greater than $\ep=\frac12$. By Lebesgue Criterion this implies that $D(x)$ is not Riemann integrable.
            \item Since $T(x)$ is only discontinuous when $x\in\Q$, and $\Q$ is a countable set, thus is also a null set. Hence, by definition of null set and continuous almost everywhere, we conclude $T(x)$ is Riemann integrable by Lebesgue Criterion.
            \item Similarly, we can also see that $H(x)$ is only discontinuous when $x=\frac1n$ for some $n\in\N$ or $x=0$ (by definition of floor function). Since $\{\frac1n\}_{n\in\N}$ is a countable set, and $\{0\}$ is finite, the union of these sets is countable thus a null set. Hence, $H(x)$ is contnuous almost everywhere and is Riemann integrable by Lebesgue Criterion.
            \item $G(x)$ is also discontinuous whenever $x=\frac1n$ for some $n\in\N$ or $x=0$ (by observering the values that $\sin\bra{\frac{\pi}{x}}$ changes its sign), same as $H(x)$, we may conclude that $G(x)$ is Riemann integrable by Lebesgue Criterion.
            \item $\ln(\frac1x)=-\ln(x)$ is also continuous everywhere except at $x=0$. So, by MAT157 since $\sin x$ is continuous everywhere, we know $\sin\bra{\ln\bra{\frac1x}}$ is continuous everywhere on $(0,1]$, thus it can be discontinuous at most at $x=0$ which is a null set. Hence, $\sin\bra{\ln\bra{\frac1x}}$ is Riemann integrable by Lebesgue Criterion.
        \end{enumerate}
    }
}

\tcbcnt{lemma}{0}
\newpage
\newq{3}{
    Let $f,g:[a,b]\to[a,b]$. Fill in the blanks below regarding the integrability of $g\circ f$ and justify your answers, by giving either a proof or a counter-example.
    $$\begin{array}{c|c|c}&f\in\mathcal{C}[a,b]&f\in\mathfrak{R}[a,b]\\\hline g\in\mathcal{C}[a,b]&\T{Yes}&\T{Yes}\\\hline g\in\mathfrak{R}[a,b]&\T{No}&\T{No}\\\hline\end{array}$$

    \begin{center}
        Table 1: Integrability of $g\circ f$ under different assumptions.
    \end{center}
    \qbreak
    \newp{\hfill

        \begin{enumerate}
            \item $f\in \mathcal{C}[a,b]$:
            \begin{enumerate}
                \item $g\in \mathcal{C}[a,b]$: Since the composition of continuous functions is continuous, so $g\circ f$ is continuous everywhere. Thus, by Lebesgue Criterion, $g\circ f$ is Riemann integrable.

                \item $g\in \mathfrak{R}[a,b]$: See below.

            \end{enumerate}
            \item $f\in \mathfrak{R}[a,b]$:
            \begin{enumerate}
                \item $g\in \mathcal{C}[a,b]$: Let $x\in[a,b]$ be such that $V_f(x)=0$, then $g\circ f$ is also continuous at $x$. Since such $x$ are almost everywhere, we conclude $g\circ f$ is continuous almost everywhere and thus Riemann integrable.
                \item $g\in \mathfrak{R}[a,b]$: Consider the example: $$f(x)=\begin{cases}\frac1q,&x=\frac{p}q\in\Q,q>0,\gcd(p,q)=1\\ 0 ,& \T{otherwise}\end{cases}, g(x)=\begin{cases}0,&x\leq 0\\ 1,&x>0\end{cases}.$$

                Then, we can see that both $f,g\in\mathfrak{R}[a,b]$, but $(g\circ f)(x)=\begin{cases}1, &x\in\Q\\ 0, & x\notin \Q\end{cases},$ which is not integrable (the Dirichlet function, as shown in the previous homework). 

            \end{enumerate}

        \end{enumerate}
    }

    \newm{
        For $1.b.$, we show $g\circ f$ is false by constructing a counter-example:

        Let $[a,b]=[0,1]$. We want to show that for some $g\in\mathfrak{R}[0,1]$, $f\in\cal{C}[0,1]$, $g\circ f$ is not Riemann integrable. To this end, we first define \[g:[0,1]\to[0,1], g(y)=\begin{cases}
            1, & y\neq0\\
            0, & y=0
        \end{cases}.\]

        Then, we want to construct a function $f$ that is both continuous on $[0,1]$ and has uncountably disconnected many points $x\in[0,1]$ such that $f(x)=0$, or $f(x)=1$ seperately. (so that $g\circ f$ is discontinuous uncountably many points thus does not satisfy Lebesgue Criterion for Riemann Integrability). So, for simplicity we will construct the case when $f(x)=0$ based on the fat cantor set (the Smith–Volterra–Cantor set) $FC$. 

        Consider the recursively defined set $FC$ as follows:

        $FC_0=[0,1]$.

        1. We take out $(\frac38,\frac58), \ie \frac14$ from the middle of $FC_0$: $FC_1=[0,\frac38]\cup[\frac58,1]$ (the length of $(\frac38,\frac58)$ is same as $[\frac38,\frac58]$ due to $\{\frac38,\frac58\}$ is a null set / measure zero).

        2. For each interval in $FC_1$, we take out the middle $\frac1{16}$ of each interval: $FC_2=[0,\frac5{32}]\cup[\frac7{32},\frac38]\cup[\frac58,\frac{25}{32}]\cup[\frac{27}{32},1]$.

        $\vdots$

        $n$. For each interval in $FC_{n-1}$, we take out the middle $\frac1{4^{n}}$ of each interval (totally $2^{n-1}$ such intervals), the remaining set is $FC_{n}$.

        In this way if we let $FC=\bigcap_{n=0}^\infty FC_n$, then $FC$ is the fat cantor set.

        We can verify the following properties of $FC$:

        
        \newl{2}{
            The `length' of $FC$ on [0,1] is $\frac12$.

            \newp{
                We consider the length of the intervals removed at each step of the construction. At the $n$-th step, the length of $2^{n-1}$ intervals removed is $\frac{2^{n-1}}{4^n}=\frac1{2^{n+1}}$, thus the total length of the intervals removed is $\sum_{n=1}^\infty\frac1{2^{n+1}}=\frac12$, which implies the length of $FC$ is $\frac12$.
            }
        }

        \newl{3}{
            $FC$ is totally disconnected and is closed.
        
            \newp{
                Let $x,y\in FC$ be arbitrary such that $x\neq y$, w.l.o.g. we let $x<y$. Moreover all points in $FC$ are endpoints of the intervals in the construction thus so are $x,y$.
        
                To obtain a contradiction, assume $x$ and $y$ are connected, that is, $[x,y]\subseteq FC$. However, by our construction of $[x,y]\subseteq FC$, such $[x,y]$ always has to take out a middle interval from $[x,y]$ by some positive length interval to get a new set $FC'$ such that $FC'\subsetneq FC$, which contradicts the fact that $FC$ is the intersection of all $FC_n$. Thus, $FC$ is totally disconnected.
        
                Moreover, since $FC$ is constructed by taking out open intervals from $[0,1]$, this implies $FC$ is closed.
            }
        }


        Now, we construct $f(x)$ as follows: 
        \[f(x)=\begin{cases}
            0, & x\in FC\\
            -(x-x_1)(x-x_2), & x\notin FC \T{ where } x\in(x_1,x_2)\subseteq[0,1]\setminus FC \st x_1,x_2\in FC
        \end{cases}.\]

        \newcl{1}{
            $f$ is defined for all $x\in[0,1]$.

            \newp{
                It suffices to show that whenever $x\notin FC$, there always exists an open interval $(x_1,x_2)$ such that $x_1,x_2\in FC$ and $(x_1,x_2)\subseteq[0,1]\setminus FC$.

                By our construction of $FC$, if $x\notin FC$, this implies there exists an open interval $(x_1,x_2)$ such that this entire open interval is `removed' from $FC$, thus $(x_1,x_2)\subseteq[0,1]\setminus FC$, showing all the middle points are also removed from $FC$. 
                
                Moreover, since by our construction, we can see the boundary / end points of $FC$ are all in $FC$ (we are always keeping the endpoints from the previous generation), thus we have $x_1,x_2\in FC$.

                Since both conditions must be satisfied when $x\notin FC$, we conclude that $f(x)$ is defined for all $x\in[0,1]$.
                
                
            }
        }

        \newcl{2}{
            $f$ is continuous on $[0,1]$.

            \newp{
                We consider the cases when $x\in FC$ and $x\notin FC$ separately.

                1. When $x\in FC$, since by our Lemma 2 $FC$ is totally disconnected and is closed, this implies there exists $x_1,x_3\in FC$ such that $x_1<x<x_3$, and $(x_1,x)\subseteq[0,1]\setminus FC$, $(x,x_3)\subseteq[0,1]\setminus FC$, $x_1,x,x_3\in FC$. Then, we can see the left limit of $f(x)$ is $\lim_{x'\to x^-}f(x')=-(x'-x_1)(x'-x)=0$ and the right limit of $f(x)$ is $\lim_{x'\to x^+}f(x')=-(x'-x)(x'-x_3)=0$, thus since both the limit of $f(x)$ is 0 and $f(x)=0$, we conclude $f(x)$ is continuous at $x$.

                2. When $x\notin FC$, we have $f(x)=-(x-x_1)(x-x_2)$ for some $x_1,x_2\in FC$ such that $x\in(x_1,x_2)\subseteq[0,1]\setminus FC$. Then, since $(x_1,x_2)$ is open, we can always find an open neighborhood of $x$ such that $f(x')=-(x'-x_1)(x'-x_2)$ for all $x'\in I_\delta (x)$, since polynomial is continuous everywhere by MAT157, we conclude $f(x)$ is continuous at $x$ locally.

                Since $x\in[0,1]$ is arbitrary, we conclude that $f$ is continuous on $[0,1]$.
            }
        }

        \newcl{3}{
            $f([0,1])\subseteq[0,1]$ (so that $f$ is a function $f:[0,1]\to[0,1]$).

            \newp{
                If $x\in FC$, then $f(x)=0\in[0,1]$.

                If $x\notin FC$, then there exist $x_1,x_2\in FC$ such that $x\in(x_1,x_2)\subseteq[0,1]\setminus FC$. Now, we can see \begin{align*}
                    f(x)&=-(x-x_1)(x-x_2),\\
                    \alt{since this parabola achieves its maximum at the midpoint of the interval, so we have:}\\
                    &\leq-\bra{\frac{x_1+x_2}2-x_1}\bra{\frac{x_1+x_2}2-x_2}\\
                    &=-\bra{\frac{x_2-x_1}{2}}\bra{\frac{x_1-x_2}{2}}\\
                    &=\frac14\bra{x_2-x_1}^2\\
                    &\leq1.
                \end{align*}

                Also $x_1<x, x<x_2$ imply $f(x)=-(x-x_1)(x-x_2)\geq0$, thus $f(x)\in[0,1]$.

                Since $x\in[0,1]$ is arbitrary, we conclude $f([0,1])\subseteq[0,1]$.
            }
        }

        \newcl{4}{
            $f(x)=0$ if and only if $x\in FC$.

            \newp{
                The backward direction holds by our definition of $f$.

                For the forward direction, we will prove the contrapositive. Assume $x\notin FC$, then by our construction of $f$ we have $f(x)=-(x-x_1)(x-x_2)$ for some $x_1,x_2\in FC$ such that $x\in(x_1,x_2)\subseteq[0,1]\setminus FC$. Then, $x\neq x_1, x\neq x_2$ imply $f(x)=-(x-x_1)(x-x_2)\neq0$, thus $f(x)\neq0$ which shows the contrapositive of the forward direction holds.

                Hence we conclude $f(x)=0$ if and only if $x\in FC$.
            }
        }


        Now, consider $g\circ f$, by Claim 4 we have $$(g\circ f)(x)=\begin{cases}0, &x\in FC\\ 1, & x\notin FC\end{cases}.$$

        Since by Lemma 1 $FC$ has `length' $\frac12$, to show it does not satisfy Lebesgue Criterion it suffices to show the discontinuous points of $g\circ f$ do not form a null set. Namely, $$\exists\ep>0,\forall \{(a_i,b_i)\}_{i\in\N}\T{ we have }\bra{FC\not\subseteq\bigcup_{i\in\N}(a_i,b_i)}\lor\bra{\sum_{i=1}^\infty(b_i-a_i)\geq\ep},$$ which is equivalent to $$\exists\ep>0,\forall \{(a_i,b_i)\}_{i\in\N}\T{ we have }\bra{FC\subseteq\bigcup_{i\in\N}(a_i,b_i)}\implies\bra{\sum_{i=1}^\infty(b_i-a_i)\geq\ep}.$$

        Since $FC$ is totally disconnected, we can see the set $FC$ contains the discontinuous points of $g\circ f$ (all points in $FC$ are also discontinuous points of $g\circ f$), thus it is enough to show $FC$ does not form a null set.

        So, fix $\ep=\frac18>0$. Let $\{(a_i,b_i)\}_{i\in\N}$ be an arbitrary open cover of $FC$. Since $FC$ has a total length of $\frac12$, and we know the total length of the cover is at least the length of $FC$, \ie $\frac12$, thus we have $\sum_{i=1}^\infty(b_i-a_i)\geq\frac12\geq\frac18=\ep$.

        Since our open cover is arbitrary, and we have constructed such $\ep>0$, we conclude that $g\circ f$ does not satisfy Lebesgue Criterion, and thus is not Riemann integrable. Moreover, since $f$ is a continuous function from $[0,1]$ to $[0,1]$ as shown in Claim 2 and Claim 3, and $g:[0,1]\to[0,1]$ is Riemann integrable, we thus found a counter-example to show that $g\circ f$ is not Riemann integrable when $f\in\cal{C}[a,b]$ and $g\in\mathfrak{R}[a,b]$.
    }
}


\end{document}  

